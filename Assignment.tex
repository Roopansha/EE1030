%iffalse
\let\negmedspace\undefined
\let\negthickspace\undefined
\documentclass[journal,12pt,twocolumn]{IEEEtran}
\usepackage{cite}
\usepackage{amsmath,amssymb,amsfonts,amsthm}
\usepackage{algorithmic}
\usepackage{graphicx}
\usepackage{textcomp}
\usepackage{xcolor}
\usepackage{txfonts}
\usepackage{listings}
\usepackage{enumitem}
\usepackage{mathtools}
\usepackage{gensymb}
\usepackage{comment}
\usepackage[breaklinks=true]{hyperref}
\usepackage{tkz-euclide} 
\usepackage{listings}
\usepackage{gvv}                                        
%\def\inputGnumericTable{}                                 
\usepackage[latin1]{inputenc}                                
\usepackage{color}                                            
\usepackage{array}                                            
\usepackage{longtable}                                       
\usepackage{calc}                                             
\usepackage{multirow}                                         
\usepackage{hhline}                                           
\usepackage{ifthen}                                           
\usepackage{lscape}
\usepackage{tabularx}
\usepackage{array}
\usepackage{float}


\newtheorem{theorem}{Theorem}[section]
\newtheorem{problem}{Problem}
\newtheorem{proposition}{Proposition}[section]
\newtheorem{lemma}{Lemma}[section]
\newtheorem{corollary}[theorem]{Corollary}
\newtheorem{example}{Example}[section]
\newtheorem{definition}[problem]{Definition}
\newcommand{\BEQA}{\begin{eqnarray}}
\newcommand{\EEQA}{\end{eqnarray}}
\newcommand{\define}{\stackrel{\triangle}{=}}
\theoremstyle{remark}
\newtheorem{rem}{Remark}

% Marks the beginning of the document
\begin{document}
\bibliographystyle{IEEEtran}
\vspace{3cm}

\title{18.Definite Integrals and Applications of Integrals}
\author{AI24BTECH11006-Bugada Roopansha}

\maketitle{Section-A JEE Advanced/IIT-JEE}
\\\\{C. MCQs with One Correct Answer}    
\\\\\\
\begin{enumerate}[start=21]
\item If $l$($m$,$n$)$=\int_{0}^{1}t^m(1+t)^n$dt, then the expression for $l$($m$,$n$) in terms of $l$($m$+1,$n$-1)is
\hspace*{\fill}\brak{2003S}
\\(a) $\frac{2^n}{m+1}$-$\dfrac{n}{m+1}$$l$($m$+1,$n$-1)
\\\\(b) $\frac{n}{m+1}$$l$($m$+1,$n$-1)
\\\\(c) $\frac{2^n}{m+1}$+$\dfrac{n}{m+1}$$l$($m$+1,$n$-1)
\\\\(d) $\frac{m}{n+1}$$l$($m$+1,$n$-1)
\\\\
\item If f(x)= $\int_{x^2}^{x^2+1}e^{-t^2}$dt ,then f(x)increases in
\hspace*{\fill}\brak{2003S}
\\(a)  (-2,2)
 \qquad       (b)  no value of x
\\\\(c)  (0,$\infty$)
\qquad(d)  (-$\infty$,0)
\\\\
\item The area bounded by the curves  $y$=$\sqrt{x}$,2$y$+3=$x$ and x-axis in the 1$^{st}$quadrant is
\hspace*{\fill}\brak{2003S}
\\(a) 9\quad(b) $\frac{27}{4}$\quad(c) 36\quad(d) 18
\\\\
\item If f(x)is differentiable and$\int_{0}^{t^2}xf(x)dx=\frac{2}{5}t^5, then f(\frac{4}{25})$ equals
\hspace*{\fill}\brak{2004S}
\\(a)$ \frac{2}{5}\qquad(b) \frac{-5}{2}\qquad(c) 1\qquad(d) \frac{5}{2}$
\\\\
\item The value of the integral$\int_{0}^{1}\sqrt{\frac{1-x}{1+x}}dx$ is
\hspace*{\fill}\brak{2004S}
\\(a)$\frac{\pi}{2}+1$\quad(b)$\frac{\pi}{2}-1$\quad(c)-1\quad(d)1
\\\\
\item The area enclosed between the curves $y$=$ax^2$and$x=ay^2$($a>0$)is 1 sq.unit,then the value of a is
\hspace*{\fill}\brak{2004S}
\\(a)$\frac{1}{\sqrt{3}}$\quad(b)$\frac{1}{2}$\quad(c)1\quad
(d)$\frac{1}{3}$
\\\\
\item $\int_{-2}^{0}$${x^3}+3{x^2}+3x+3+(x+1)cos(x+1)$dx is equal to
\hspace*{\fill}\brak{2005S}
\\(a)-4\quad(b)0\quad(c)4\quad(d)6
\\\\
\item The area bounded by the parabolas $y=(x+1)^2$and$y=(x-1)^2$and the line$y=\frac{1}{4}$is
\hspace*{\fill}\brak{2005S}
\\(a) 4 sq.units\quad(b)$ \frac{1}{6}sq.units$ \\
(c)$\frac{4}{3}$sq.units\quad(d)$\frac{1}{3}$sq.units
\\\\
\item The area of the region between the curves $y=\sqrt{\frac{1+sinx}{cosx}}$and$y=\sqrt{\frac{1-sinx}{cosx}}$bounded by the lines $x=0$and$x=\frac{\pi}{4}$is
\hspace*{\fill}\brak{2008}
\\(a)$\int_{0}^{\sqrt{2}-1}\frac{t}{(1+{t^2})\sqrt{1-{t^2}}}dt$\qquad(b)$\int_{0}^{\sqrt{2}-1}$                     
$\frac{4t}{(1+{t^2})\sqrt{1-{t^2}}}$dt\\
(c)$\int_{0}^{\sqrt{2}+1}\frac{4t}{(1+{t^2})\sqrt{1-{t^2}}}dt$
\qquad
(d)$\int_{0}^{\sqrt{2}+1}\frac{t}{(1+{t^2})\sqrt{1-{t^2}}}dt$
\\\\
\item Let $f$be a non-negative function defined on the interval [$0$,1].If$\int_{0}^{x}\sqrt{1-(f'(t))^2}=\int_{0}^{x}f(t)dt,$$    0\leq x \leq 1,$and$f(0)=0,then
\hspace*{\fill}\brak{2009}
\\(a)f(\frac{1}{2})$$<\frac{1}{2}$and$f(\frac{1}{3})>\frac{1}{3}$
\\(b)$f(\frac{1}{2})>\frac{1}{2}andf(\frac{1}{3})>\frac{1}{3}$
\\(c)$f(\frac{1}{2})<\frac{1}{2}andf(\frac{1}{3})<\frac{1}{3}$
\\(d)$f(\frac{1}{2})>\frac{1}{2}andf(\frac{1}{3})<\frac{1}{3}$
\\\\
\item The value of $\lim_{x\to0}\frac{1}{x^3}\int_{0}^{x}\frac{t\ln(1+t)}{t^4 +4}dt$ is
\hspace*{\fill}\brak{2010}
\\(a) 0\qquad(b)$\frac{1}{12}\qquad(c)\frac{1}{24}\qquad(d)\frac{1}{64}$
\\\\
\item Let $f$ be a real valued function defined on the interval (-1,1) such that $e^{-x}f(x)=2+\int_{0}^{x}\sqrt{t^4 +1}dt$, for all $x\in(-1,1)$,and let $f^{-1}$be the inverse function $f$.Then $(f^{-1})'(2)$is equal to
\hspace*{\fill}\brak{2010}
\\(a) 1\qquad (b)$\frac{1}{3}\qquad (c)\frac{1}{2}\qquad(d)\frac{1}{e}$
\\\\
\item The value of $\int_{\sqrt{\ln 2}}^{\sqrt{\ln 3}}\frac{xsinx^2}{sinx^2 + sin(\ln 6-x^2)}dx$ is
\hspace*{\fill}\brak{2011}
\\(a)$\frac{1}{4}\ln \frac{3}{2}\qquad(b)\frac{1}{2}\ln\frac{3}{2}\qquad(c)\ln \frac{3}{2}\qquad (d)\frac{1}{6}\ln \frac{3}{2}$
\\\\
\item Let the straight line$x=b$ divide the area enclosed by $y=(1-x)^2$,$y=0$,and$x=0$into two parts $R_1 (0\leq x\leq b)$ and $R_2(b\leq x\leq 1)$ such that $R_1 -R_2 =\frac{1}{4}.$Then  
$b$ equals
\hspace*{\fill}\brak{2011}
\\(a)$\frac{3}{4}\qquad(b)\frac{1}{2}\qquad(c)\frac{1}{3}\qquad(d)\frac{1}{4}$
\\\\
\item Let $f:[-1,2]\to [0,\infty)$ be a continuous function such that $f(x)=f(1-x)$for all $x\in [-1,2]$.Let $R_1=\int_{-1}^{2}xf(x)dx$, and $R_2 $be the area of the region bounded by $y=f(x) ,x=-1,x=2,$and the x-axis.\\Then
\hspace*{\fill}\brak{2011}
\\(a)$R_1=2R_2$\qquad(b)$R_1=3R_2$
\\(c)$2R_1=R_2$\qquad (d)$3R_1=R_2$






\end{enumerate}



\renewcommand{\thefigure}{\theenumi}
\renewcommand{\thetable}{\theenumi}


\end{document}
