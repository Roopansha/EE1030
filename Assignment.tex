%iffalse
\let\negmedspace\undefined
\let\negthickspace\undefined
\documentclass[journal,12pt,twocolumn]{IEEEtran}
\usepackage{cite}
\usepackage{amsmath,amssymb,amsfonts,amsthm}
\usepackage{algorithmic}
\usepackage{graphicx}
\usepackage{textcomp}
\usepackage{xcolor}
\usepackage{txfonts}
\usepackage{listings}
\usepackage{enumitem}
\usepackage{mathtools}
\usepackage{gensymb}
\usepackage{comment}
\usepackage[breaklinks=true]{hyperref}
\usepackage{tkz-euclide} 
\usepackage{listings}
\usepackage{gvv}                                        
%\def\inputGnumericTable{}                                 
\usepackage[latin1]{inputenc}                                
\usepackage{color}                                            
\usepackage{array}                                            
\usepackage{longtable}                                       
\usepackage{calc}                                             
\usepackage{multirow}                                         
\usepackage{hhline}                                           
\usepackage{ifthen}                                           
\usepackage{lscape}
\usepackage{tabularx}
\usepackage{array}
\usepackage{float}


\newtheorem{theorem}{Theorem}[section]
\newtheorem{problem}{Problem}
\newtheorem{proposition}{Proposition}[section]
\newtheorem{lemma}{Lemma}[section]
\newtheorem{corollary}[theorem]{Corollary}
\newtheorem{example}{Example}[section]
\newtheorem{definition}[problem]{Definition}
\newcommand{\BEQA}{\begin{eqnarray}}
\newcommand{\EEQA}{\end{eqnarray}}
\newcommand{\define}{\stackrel{\triangle}{=}}
\theoremstyle{remark}
\newtheorem{rem}{Remark}

% Marks the beginning of the document
\begin{document}
\bibliographystyle{IEEEtran}
\vspace{3cm}

\title{18.Definite Integrals and Applications of Integrals}
\author{AI24BTECH11006-Bugada Roopansha}

\maketitle{Section-A JEE Advanced/IIT-JEE}
\\\\{C. MCQs with One Correct Answer}    
\begin{enumerate}[start=21]
\item If $l$$\brak{m,n}$=$\int_{0}^{1}t^m\brak{1+t}^n$dt, then the expression for $l$ $\brak{m,n}$ in terms of $l$$\brak{m+1,n-1}$is
\hfill{(2003S)}
\begin{enumerate}
\item $\frac{2^n}{m+1}$-$\frac{n}{m+1}$$l$$\brak{m+1,n-1}$
\item $\frac{n}{m+1}$$l$$\brak{m+1,n-1}$
\item $\frac{2^n}{m+1}$+$\frac{n}{m+1}$$l$$\brak{m+1,n-1}$
\item $\frac{m}{n+1}$$l\brak{m+1,n-1}$
\end{enumerate}
\item If f\brak{x}= $\int_{x^2}^{x^2+1}e^{-t^2}$dt ,then f\brak{x}increases in\\
\hfill{(2003S)}
\begin{enumerate}
\item  $\brak{-2,2}$
\item no value of x
\item  $\brak{0,\infty}$
\item  $\brak{-\infty,0}$
\end{enumerate}
\item The area bounded by the curves  $y$=$\sqrt{x}$,$2$y$+3$=$x$ and x-axis in the $1^{st}$quadrant is
\hfill{(2003S)}
\begin{enumerate}
\item $9$
\item$\frac{27}{4}$
\item $36$
\item $18$
\end{enumerate}
\item If f\brak{x}is differentiable and$\int_{0}^{t^2}$$xf\brak{x}dx=\frac{2}{5}t^5$, then $f\brak{\frac{4}{25}}$ equals
\hfill{(2004S)}
\begin{enumerate}
\item $\frac{2}{5}$
\item $\frac{-5}{2}$
\item  $1$
\item $\frac{5}{2}$
\end{enumerate}
\item The value of the integral$\int_{0}^{1}\sqrt{\frac{1-x}{1+x}}dx$ is
\hfill{(2004S)}
\begin{enumerate}
\item$\frac{\pi}{2}+1$
\item$\frac{\pi}{2}-1$
\item$-1$
\item$1$
\end{enumerate}
\item The area enclosed between the curves 
$y$=$ax^2$and$x=ay^2$$\brak{a>0}$is $1 sq.unit$,then the value of a is
\hfill{(2004S)}
\begin{enumerate}
\item$\frac{1}{\sqrt{3}}$
\item$\frac{1}{2}$
\item$1$
\item$\frac{1}{3}$
\end{enumerate}
\item $\int_{-2}^{0}$${x^3}+3{x^2}+3x+3+\brak{x+1}cos\brak{x+1}$dx is equal to
\hfill{(2005S)}
\begin{enumerate}
\item$-4$
\item$0$
\item$4$
\item$6$
\end{enumerate}
\item The area bounded by the parabolas $y=\brak{x+1}^2$and$y=\brak{x-1}^2$and the line$y=\frac{1}{4}$is
\hfill{(2005S)}
\begin{enumerate}
\item $4 sq.units$
\item$ \frac{1}{6}sq.units$
\item$\frac{4}{3}$sq.units
\item$\frac{1}{3}$sq.units
\end{enumerate}
\item The area of the region between the curves $y=\sqrt{\frac{1+sinx}{cosx}}$and$y=\sqrt{\frac{1-sinx}{cosx}}$bounded by the lines $x=0$and$x=\frac{\pi}{4}$is
\hfill{(2008)}
\begin{enumerate}
\item$\int_{0}^{\sqrt{2}-1}\frac{t}{\brak{1+{t^2}}\sqrt{1-{t^2}}}dt$
\item$\int_{0}^{\sqrt{2}-1}$$\frac{4t}{\brak{1+{t^2}}\sqrt{1-{t^2}}}$dt
\item$\int_{0}^{\sqrt{2}+1}\frac{4t}
{\brak{1+{t^2}}\sqrt{1-{t^2}}}dt$
\item$\int_{0}^{\sqrt{2}+1}\frac{t}{\brak{1+{t^2}}\sqrt{1-{t^2}}}dt$
\end{enumerate}
\item Let $f$be a non-negative function defined on the interval $\sbrak{0,1}$$\cdot$If$\int_{0}^{x}\sqrt{1-\brak{f'\brak{t}}^2}=\int_{0}^{x}f\brak{t}dt,$$    0\leq x \leq 1,$and$f\brak{0}=0$,then
\hfill{(2009)}
\begin{enumerate}
\item $f\brak{\frac{1}{2}}$$<\frac{1}{2}$and$f\brak{\frac{1}{3}}>\frac{1}{3}$
\item$f\brak{\frac{1}{2}}>\frac{1}{2}andf\brak{\frac{1}{3}}>\frac{1}{3}$
\item$f\brak{\frac{1}{2}}<\frac{1}{2}andf\brak{\frac{1}{3}}<\frac{1}{3}$
\item $f\brak{\frac{1}{2}}>\frac{1}{2}andf\brak{\frac{1}{3}}<\frac{1}{3}$
\end{enumerate}
\item The value of $\displaystyle\lim_{x\to0}\frac{1}{x^3}\int_{0}^{x}\frac{t\ln\brak{1+t}}{t^4 +4}dt$ is
\hfill{(2010)}
\begin{enumerate}
\item $0$
\item $\frac{1}{12}$
\item$\frac{1}{24}$\item$\frac{1}{64}$
\end{enumerate}
\item Let $f$ be a real valued function defined on the interval \brak{-1,1} such that $e^{-x}f\brak{x}=2+\int_{0}^{x}\sqrt{t^4 +1}dt$, for all $x\in\brak{-1,1}$,and let $f^{-1}$be the inverse function $f$.Then $\brak{f^{-1}}'(2)$is equal to
\hfill{(2010)}
\begin{enumerate}
\item $1$
\item$\frac{1}{3} $
\item $\frac{1}{2}$
\item $\frac{1}{e}$
\end{enumerate}
\item The value of $\int_{\sqrt{\ln 2}}^{\sqrt{\ln 3}}\frac{xsinx^2}{sinx^2 + sin\brak{\ln 6-x^2}}dx$ is
\hfill{(2011)}
\begin{enumerate}
\item $\frac{1}{4}\ln \frac{3}{2}$
\item$\frac{1}{2}\ln\frac{3}{2}$
\item $\ln \frac{3}{2}$ 
\item $\frac{1}{6}\ln \frac{3}{2}$
\end{enumerate}
\item Let the straight line$x=b$ divide the area enclosed by $y=\brak{1-x}^2$,$y=0$,and$x=0$into two parts $R_1 \brak{0\leq x\leq b}$ and $R_2\brak{b\leq x\leq 1}$ such that $R_1 -R_2 =\frac{1}{4}\cdot$Then  
$b$ equals
\hfill{(2011)}
\begin{enumerate}
\item$\frac{3}{4}$
\item $\frac{1}{2}$
\item $\frac{1}{3}$
\item $\frac{1}{4}$
\end{enumerate}
\item Let $f:\sbrak{-1,2}\to [0,\infty)$ be a continuous function such that $f\brak{x}=f\brak{1-x}$for all $x\in \brak{-1,2}$.Let $R_1=\int_{-1}^{2}xf\brak{x}dx$, and $R_2 $be the area of the region bounded by $y=f\brak{x}$ ,$x=-1$,$x=2$,and the x-axis $\cdot$\\Then
\hfill{(2011)}
\begin{enumerate}
\item $R_1=2R_2$
\item $R_1=3R_2$
\item $2R_1=R_2$ 
\item $3R_1=R_2$
\end{enumerate}






\end{enumerate}



\renewcommand{\thefigure}{\theenumi}
\renewcommand{\thetable}{\theenumi}


\end{document}
