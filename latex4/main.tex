%iffalse
\let\negmedspace\undefined
\let\negthickspace\undefined
\documentclass[journal,12pt,twocolumn]{IEEEtran}
\usepackage{cite}
\usepackage{amsmath,amssymb,amsfonts,amsthm}
\usepackage{algorithmic}
\usepackage{graphicx}
\usepackage{textcomp}
\usepackage{xcolor}
\usepackage{txfonts}
\usepackage{listings}
\usepackage{enumitem}
\usepackage{mathtools}
\usepackage{gensymb}
\usepackage{comment}
\usepackage[breaklinks=true]{hyperref}
\usepackage{tkz-euclide} 
\usepackage{listings}
\usepackage{gvv}                                        
%\def\inputGnumericTable{}    
\renewcommand{\thefigure}{\theenumi}
\renewcommand{\thetable}{\theenumi}
\usepackage[latin1]{inputenc}                                
\usepackage{color}                                            
\usepackage{array}                                            
\usepackage{longtable}                                       
\usepackage{calc}                                             
\usepackage{multirow}                                         
\usepackage{hhline}                                           
\usepackage{ifthen}                                           
\usepackage{lscape}
\usepackage{tabularx}
\usepackage{array}
\usepackage{float}
\newtheorem{theorem}{Theorem}[section]
\newtheorem{problem}{Problem}
\newtheorem{proposition}{Proposition}[section]
\newtheorem{lemma}{Lemma}[section]
\newtheorem{corollary}[theorem]{Corollary}
\newtheorem{example}{Example}[section]
\newtheorem{definition}[problem]{Definition}
\newcommand{\BEQA}{\begin{eqnarray}}
\newcommand{\EEQA}{\end{eqnarray}}
\theoremstyle{remark}
% Marks the beginning of the document
\begin{document}
\bibliographystyle{IEEEtran}
\vspace{3cm}
\title{2024-January Session-01-29-2024-shift-1}
\author{AI24BTECH11006}
\maketitle
\section{SECTION - A}
\begin{enumerate}[start=1]
\item Let a die be rolled until a $2$ is obtained. The probability that a $2$ is obtained on an even-numbered toss is equal to:
\hfill{\sbrak{January-2024}}
	\begin{enumerate}
    \item $\frac{5}{11}$
    \item $\frac{5}{6}$
    \item $\frac{1}{11}$
    \item $\frac{6}{11}$
\end{enumerate}
\item $\lim_{x \to \frac{\pi^-}{2}} \frac{\int_{x^3}^{\brak{\frac{\pi}{2}}^2} \cos t^\frac{1}{3} dt}{\brak{x-\frac{\pi}{2}}^2} $
\hfill{\sbrak{January-2024}}
	\begin{enumerate}
    \item $\frac{3 \pi^2}{4}$
    \item $\frac{3 \pi}{4}$
    \item $\frac{3 \pi^2}{8}$
    \item $\frac{3 \pi}{8}$
\end{enumerate} 
\item Consider the equation $4\sqrt{2}x^3 - 3\sqrt{2}x - 1 = 0$. \\
Statement 1: The solution of this equation is $\cos \frac{\pi}{12}$. \\
Statement 2: This equation has only one real solution. 
\hfill{\sbrak{January-2024}}
		\begin{enumerate}
    \item Both statements are true.
    \item Statement $1$ is true but Statement $2$ is false.
    \item Statement $1$ is false but Statement $2$ is true.
    \item Both statements are false.
\end{enumerate}
\item If $\mod{2A}^3 = 2^{21}$ and $
A = \begin{bmatrix}
1 & 0 & 0 \\
0 & \alpha & \beta \\
0 & \beta & \alpha 
\end{bmatrix}$
, then $\alpha$ is:
\hfill{\sbrak{January-2024}}
		\begin{enumerate}
    \item $5$
    \item $3$
    \item $9$
    \item $17$
\end{enumerate}
\item In a GP with $64$ terms, if the sum of all terms is seven times the sum of the odd terms, the common ratio is:
\hfill{\sbrak{January-2024}}
	\begin{enumerate}
    \item $3$
    \item $4$
    \item $5$
    \item $6$
\end{enumerate}
\item Given $\frac{dy}{dx} -\brak{ \frac{\sin 2x}{1 + \cos^2 x}}y=\brak{\frac{\sin x}{1+\cos^2x}}$ and $y(0) = 0$, find $y \brak{ \frac{\pi}{2}}$.
\hfill{\sbrak{January-2024}}
	\begin{enumerate}
    \item $-1$
    \item $1$
    \item $0$
    \item $2$
\end{enumerate}
\item  $4 \cos \theta + 5 \sin \theta = 1$ . Then find $\tan \theta$, where $\theta \in \left( -\frac{\pi}{2}, \frac{\pi}{2} \right)$.
\hfill{\sbrak{January-2024}}
	\begin{enumerate}
    \item $ \frac{10-\sqrt{10}}{6}$
        \item $ \frac{10-\sqrt{10}}{12}$
    \item $ \frac{\sqrt{10}-10}{6}$
     \item $ \frac{\sqrt{10}-10}{12}$
\end{enumerate}
\item In an increasing arithmetic progression $a_1, a_2, \dots, a_n$, if $a_6 = 2$ and the product of $a_1, a_5, a_4$ is the greatest, then the common difference $d$ is:
\hfill{\sbrak{January-2024}}
	\begin{enumerate}
    \item $1.6$
    \item $1.8$
    \item $0.6$
    \item $2.0$
\end{enumerate}
\item If the relation $R: \brak{a,b} R\brak{c,d}$ holds only if $ad - bc$ is divisible by $5$ where $a,b,c,d \in \mathbb{Z}$, then $R$ is:
\hfill{\sbrak{January-2024}}
	\begin{enumerate}
    \item Reflexive
    \item Symmetric, Reflexive but not Transitive
    \item Reflexive, Transitive but not Symmetric
    \item An Equivalence Relation
\end{enumerate}
\item Let $f\brak{x}$ and $g\brak{x}$ be defined as follows:\\ 
$
f\brak{x} = 
\begin{cases}
2x + 2 & $if $ x \in \brak{-1, 0} \\
1 - \frac{x}{3} & $if $ x \in \sbrak{0, 3}
\end{cases}
$\\ 
$$
g\brak{x} =
\begin{cases}
x & \text{if } x \in \sbrak{0, 1} \\
-x & \text{if } x \in \brak{-3, 0}
\end{cases}
$$The range of $fog\brak {x}$ is:
\hfill{\sbrak{January-2024}}
\begin{enumerate}
    \item $\sbrak{0, 1}$
    \item $\sbrak{-1, 1}$
    \item $\sbrak{0, 1}$
    \item $\brak{-1, 1}$
\end{enumerate}
\item If $\int_{\frac{-\pi}{2}}^\frac{\pi}{2}\brak{ \frac{x^2 \cos x}{1 + \pi^x}+\frac{1+\sin^2x}{1+e^{\brak{\sin x}^{2023}}}} \, dx = \frac{\pi}{4}\brak{\pi + \alpha}-2$, then $\alpha$ is equal to:
\hfill{\sbrak{January-2024}}
	\begin{enumerate}
    \item $1$
    \item $2$
    \item $3$
    \item $4$
\end{enumerate}
\item The area under the curve $x^2 + y^2 = 169$ and below the line $5x - y = 13$ is:
\hfill{\sbrak{January-2024}}
	\begin{enumerate}
    \item $\frac{169 \pi}{4} - \frac{65}{2} + \frac{169}{2} \sin^{-1} \frac{12}{13}$
    \item $\frac{169 \pi}{4} + \frac{65}{2} - \frac{169}{2} \sin^{-1} \frac{12}{13}$
    \item $\frac{169 \pi}{4} - \frac{65}{2} + \frac{169}{2} \sin^{-1} \frac{13}{14}$
    \item $\frac{169 \pi}{4} + \frac{65}{2} + \frac{169}{2} \sin^{-1} \frac{13}{14}$
\end{enumerate}
\item If $f\brak{x} = \frac{\brak{2^x +2^{-x}}\brak{\tan x}\tan^{-1} \brak{2x^2 - 3x + 1}}{\brak{7x^2 - 3x +1}^3}$, then $f\prime\brak{0}$ is:
\hfill{\sbrak{January-2024}}
	\begin{enumerate}
    \item $\sqrt{\pi}$
    \item $\sqrt{\frac{\pi}{4}}$
    \item $\pi$
    \item $2\cdot\pi^\frac{3}{2}$
\end{enumerate}
\item Evaluate $\int \frac{\brak{\sin x - \cos x} \sin^2 x}{\sin x\cos^2 x + \tan x\sin^3 x} \, dx$ is equal to
\hfill{\sbrak{January-2024}}
	\begin{enumerate}
    \item $\frac{1}{3} \ln \abs{\sin^3 x - \cos^3 x} + C$
    \item $\frac{1}{3} \ln \abs{\sin^3 x + \cos^3 x} + C$
    \item $\frac{1}{2} \ln \abs{\sin^3 x - \cos^3 x} + C$
    \item $\frac{1}{4} \ln \abs{\sin^3 x + \cos^3 x} + C$
\end{enumerate}
\end{enumerate}
\end{document}
