%iffalse
\let\negmedspace\undefined
\let\negthickspace\undefined
\documentclass[journal,12pt,twocolumn]{IEEEtran}
\usepackage{cite}
\usepackage{amsmath,amssymb,amsfonts,amsthm}
\usepackage{algorithmic}
\usepackage{graphicx}
\usepackage{textcomp}
\usepackage{xcolor}
\usepackage{txfonts}
\usepackage{listings}
\usepackage{enumitem}
\usepackage{mathtools}
\usepackage{gensymb}
\usepackage{comment}
\usepackage[breaklinks=true]{hyperref}
\usepackage{tkz-euclide} 
\usepackage{listings}
\usepackage{gvv}                                        
%\def\inputGnumericTable{}    
\renewcommand{\thefigure}{\theenumi}
\renewcommand{\thetable}{\theenumi}
\usepackage[latin1]{inputenc}                                
\usepackage{color}                                            
\usepackage{array}                                            
\usepackage{longtable}                                       
\usepackage{calc}                                             
\usepackage{multirow}                                         
\usepackage{hhline}                                           
\usepackage{ifthen}                                           
\usepackage{lscape}
\usepackage{tabularx}
\usepackage{array}
\usepackage{float}
\newtheorem{theorem}{Theorem}[section]
\newtheorem{problem}{Problem}
\newtheorem{proposition}{Proposition}[section]
\newtheorem{lemma}{Lemma}[section]
\newtheorem{corollary}[theorem]{Corollary}
\newtheorem{example}{Example}[section]
\newtheorem{definition}[problem]{Definition}
\newcommand{\BEQA}{\begin{eqnarray}}
\newcommand{\EEQA}{\end{eqnarray}}
\theoremstyle{remark}
% Marks the beginning of the document
\begin{document}
\bibliographystyle{IEEEtran}
\vspace{3cm}
\title{2023-January Session-01-30-2023-shift-1}
\author{AI24BTECH11006}
\maketitle
\section{SECTION - A}
\begin{enumerate}[start=16]
\item If the solution of the equation $\log\brak{\cos{x}}\cot{x} + 4\log\brak{\sin{x}}\tan{x} = 1$, $x \in \sbrak{0, \frac{\pi}{2}}$ is $\sin^{-1}\brak{\frac{\alpha + \beta}{2}}$, where $\alpha, \beta$ are integers, then $\alpha + \beta$ is equal to 
    \hfill{\sbrak{January-2023}}
	\begin{enumerate}
    \item $6$
        \item $5$
        \item $4$
        \item $3$
    \end{enumerate}
\item A straight line cuts off the intercepts $OA = a$ and $OB = b$ on the positive direction of the x-axis and y-axis respectively. If the perpendicular from the origin $O$ to this line makes an angle of $\frac{\pi}{6}$ with the positive direction of the y-axis and the area of $\triangle OAB$ is $\frac{98}{3\sqrt{3}}$, then $a^2 - b^2$ is equal to 
   \hfill{\sbrak{January-2023}}
	\begin{enumerate}
        \item $196$
        \item $\frac{196}{3}$
        \item $\frac{392}{3}$
        \item $98$
    \end{enumerate}
\item If $a_n=\frac{-2}{4n^2 - 16n + 5}$, then $a_1 + a_2 + \dots + a_{25}$ is equal to 
   \hfill{\sbrak{January-2023}}
	\begin{enumerate}
        \item $\frac{49}{138}$
        \item $\frac{52}{147}$
        \item $\frac{51}{144}$
        \item $\frac{50}{141}$
    \end{enumerate}
\item Let the solution curve $y = y\brak{x}$ of the differential equation $\frac{dy}{dx} - \frac{3 x^5 \tan^{-1}\brak{x^3}}{\brak{1+x^6}^\frac{3}{2}}y=2x\exp{\frac{x^3-tan^{-1} x^3}{\sqrt{1+x^6}}}$ pass through the origin. Then  $y\brak{1}$ is equal to 
   \hfill{\sbrak{January-2023}}
	\begin{enumerate}
        \item $\exp\brak{\frac{1-\pi}{4\sqrt{2}}}$
        \item $\exp\brak{\frac{4-\pi}{4\sqrt{2}}}$
        \item $\exp\brak{\frac{4+\pi}{4}}$
        \item $\exp\brak{\frac{\pi-4}{2\sqrt{2}}}$
    \end{enumerate}
\item If an unbiased die, marked with $-2, -1, 0, 1, 2, 3$ on its faces, is thrown five times, then the probability that the product of the outcomes is positive is: 
   \hfill{\sbrak{January-2023}}
	\begin{enumerate}
        \item $\frac{881}{2592}$
        \item $\frac{440}{2592}$
        \item $\frac{27}{288}$
        \item $\frac{521}{2592}$
    \end{enumerate}
\section{SECTION-B}
\item Let $S = \sbrak{1, 2, 3, 4, 5, 6}$. The number of one-to-one functions $f : S \to P\brak{S}$, such that $f\brak{n} \subset f\brak{m}$ where $n \textless m$, is equal to $\cdots$
	\hfill{\sbrak{January-2023}}
\item The number of four-digit numbers $\brak{repetition of digits allowed}$ made using the digits $1, 2, 3, and 5$ and divisible by $15$, is $\cdots$
	\hfill{\sbrak{January-2023}}
\item If $\lambda_1$ \textless$ \lambda_2$ are two values of $\lambda$ such that the angle between the planes $P_1:\Bar{r}\brak{3\hat{i} - 5\hat{j} + \hat{k} }= 7$ and $P_2: \Bar{r}\brak{\lambda \hat{i} +\hat{j} - 3\hat{k}} = 9$ is 
$\sin^{-1}\brak{\frac{2\sqrt{6}}{5}}$, then the square of the length of the perpendicular from the point $\brak{38\lambda_1, 10\lambda_2, 2}$ to the plane $P_1$ is $\cdots$
\hfill{\sbrak{January-2023}}
\item Let $\sum_{n=0}^{\infty} \frac{n^3\brak{\brak{2n}!}+\brak{2n-1}\brak{n!}}{\brak{n!}\brak{\brak{2n}!}} = ae+\frac{b}{e}+c$, where $a, b, c \in \mathbb{Z}$ and $e=\sum_{n=0}^{\infty}\frac{1}{n!}$.Then $a^2 - b + c$ is equal to $\cdots$
	\hfill{\sbrak{January-2023}}
\item Let $ z = 1 + i $ and $z_1=\frac{1+i\Bar{z}}{\Bar{z}\brak{1-z}+\frac{1}{z}}$. Then $\frac{12}{\pi}\arg\brak{z_1}$ is equal to $\cdots$
	\hfill{\sbrak{January-2023}}
\item Let $f^1\brak{x} =\frac{3x+2}{2x+3} , x \in \mathbb{R}-\{\frac{-3}{2}\}$.For $ n \geq 2 $, define $f^n\brak{x} = f^1of^{n-1}\brak{x}.$If$ f^5\brak{x} = \frac{ax + b}{bx + a},$gcd$\brak{a, b} = 1$,
then  a + b  is equal to $\cdots$
\hfill{\sbrak{January-2023}}
\item $\lim_{x \to 0} \frac{48}{x^4} \int_0^{x} \frac{t^3}{t^6+1}dt$ is equal to $\cdots$
	\hfill{\sbrak{January-2023}}
\item The mean and variance of $7$ observations are $8$ and
$16$respectively. If one observation $14$ is omitted
and a and b are respectively mean and variance of
remaining $6$ observation, then $a + 3b -5$ is equal to $\cdots$
\hfill{\sbrak{January-2023}}
\item If the equation of the plane passing through the
point $\brak{1, 1, 2}$ and perpendicular to the line$\brak{ x - 3y +
2z - 1 = 0 = 4x - y + z}$ is Ax + By + Cz = 1, then
		$140\brak{C - B + A}$ is equal to $\cdots$
		\hfill{\sbrak{January-2023}}
\item Let $\alpha$ be the area of the larger region bounded by the curve $y^2 = 8x$, the line $y = x$, and $x = 2$, which lies in the first quadrant. Then the value of $3\alpha$ is equal to $\cdots$
	\hfill{\sbrak{January-2023}}





    
\end{enumerate}






\end{document}
