%iffalse
\let\negmedspace\undefined
\let\negthickspace\undefined
\documentclass[journal,12pt,twocolumn]{IEEEtran}
\usepackage{cite}
\usepackage{amsmath,amssymb,amsfonts,amsthm}
\usepackage{algorithmic}
\usepackage{graphicx}
\usepackage{textcomp}
\usepackage{xcolor}
\usepackage{txfonts}
\usepackage{listings}
\usepackage{enumitem}
\usepackage{mathtools}
\usepackage{gensymb}
\usepackage{comment}
\usepackage[breaklinks=true]{hyperref}
\usepackage{tkz-euclide} 
\usepackage{listings}
\usepackage{gvv}                                        
%\def\inputGnumericTable{}    
\renewcommand{\thefigure}{\theenumi}
\renewcommand{\thetable}{\theenumi}
\usepackage[latin1]{inputenc}                                
\usepackage{color}                                            
\usepackage{array}                                            
\usepackage{longtable}                                       
\usepackage{calc}                                             
\usepackage{multirow}                                         
\usepackage{hhline}                                           
\usepackage{ifthen}                                           
\usepackage{lscape}
\usepackage{tabularx}
\usepackage{array}
\usepackage{float}
\newtheorem{theorem}{Theorem}[section]
\newtheorem{problem}{Problem}
\newtheorem{proposition}{Proposition}[section]
\newtheorem{lemma}{Lemma}[section]
\newtheorem{corollary}[theorem]{Corollary}
\newtheorem{example}{Example}[section]
\newtheorem{definition}[problem]{Definition}
\newcommand{\BEQA}{\begin{eqnarray}}
\newcommand{\EEQA}{\end{eqnarray}}
\theoremstyle{remark}
% Marks the beginning of the document
\begin{document}
\bibliographystyle{IEEEtran}
\vspace{3cm}
\title{2024-April Session-04-04-2024-shift-2}
\author{AI24BTECH11006 - Bugada Roopansha}
\maketitle
\section{SECTION - A}
\begin{enumerate}[start=16]
\item For $\lambda > 0$ let $\theta$ be the angle between the vectors $\mathbf{a} = \hat{i} + \lambda \hat{j} - 3 \hat{k}$ and $\mathbf{b} = 3 \hat{i} - \hat{j} + 2 \hat{k}$. If the vectors $\mathbf{\Bar{a}} + \mathbf{\Bar{b}}$ and $\mathbf{\Bar{a}} - \mathbf{\Bar{b}}$ are mutually perpendicular, then the value of $\brak{14 \cos \theta}^2$ is equal to:
    \begin{enumerate}
        \item $50$
        \item $25$
        \item $20$
        \item $40$
    \end{enumerate}

    \item If the value of the integral $\int_{-1}^{1} \frac{\cos \alpha}{1 + 3^x}dx is \frac{2}{\pi}$, then a value of $\alpha$ is:
    \begin{enumerate}
        \item $\frac{\pi}{3}$
        \item $\frac{\pi}{2}$
        \item $\frac{\pi}{4}$
        \item $\frac{\pi}{6}$
    \end{enumerate}

    \item Let $\mathbf{a} = \hat{i} + \hat{j} + \hat{k}$, $\mathbf{b} = 2 \hat{i} + 4 \hat{j} - 5 \hat{k}$, and $\mathbf{c} = x \hat{i} + 2 \hat{j} + 3 \hat{k}$, where $x \in \mathbb{R}$. If $\mathbf{\Bar{d}}$ is the unit vector in the direction of $\mathbf{\Bar{b}} + \mathbf{\Bar{c}}$ such that $\mathbf{\Bar{a}} \cdot \mathbf{\Bar{d}} = 1$, then $(\mathbf{\Bar{a}} \times \mathbf{\Bar{b}}) \cdot \mathbf{\Bar{c}}$ is equal to:
    \begin{enumerate}
        \item $3$
        \item $6$
        \item $11$
        \item $9$
    \end{enumerate}

    \item Let a relation $R$ on $\mathbb{N} \times \mathbb{N}$ be defined as: $(x_1, y_1) R (x_2, y_2)$ if and only if $x_1 \leq x_2$ or $y_1 \leq y_2$. Consider the two statements:
    \begin{enumerate}
        \item $R$ is reflexive but not symmetric
        \item $R$ is transitive
    \end{enumerate}
    Which one of the following is true:
    \begin{enumerate}
        \item Both \brak{I} and \brak{II} are correct
        \item Only \brak{I} is correct
        \item Only \brak{II} is correct
        \item Neither \brak{I} nor \brak{II} is correct
    \end{enumerate}

    \item If the function:
    $$
    f(x) = 
    \begin{cases}
    \frac{72^a-9^a-8^a+1}{\sqrt{2}-\sqrt{1+\cos x}} &  x \neq 0 \\
    a\ln{2}\ln{3}, &  x = 0
    \end{cases}
    $$
    is continuous at $x = 0$, then the value of $a^2$ is:
    \begin{enumerate}
        \item $1152$
        \item $746$
        \item $968$
        \item $1250$
    \end{enumerate}

    \item There are $4$ men and $5$ women in Group A, and $5$ men and $4$ women in Group B. If $4$ persons are selected from each group, then the number of ways of selecting $4$ men and $4$ women is$\cdots$
    

    \item Consider a triangle $ABC$ having the vertices $A\brak{1, 2}$, $B\brak{\alpha, \beta}$, and $C\brak{\gamma, \delta}$ and angles $\angle ABC = \frac{\pi}{6}$ and $\angle BAC = \frac{2\pi}{3}$. If points $B$ and $C$ lie on the line $y = x + 4$, then $\alpha^2 + \gamma^2$ is equal to$\cdots$
    

    \item Let $y = y\brak{x}$ be the solution of the differential equation $\brak{x + y + 2}^2  dx = dy$, $y\brak{0} = -2$. Let the maximum and minimum values of the function $y\brak{x}$ in $\sbrak{0, \frac{\pi}{3}}$ be $\alpha$ and $\beta$, respectively. If $\brak{3\alpha + \pi}^2 + \beta^2 = \gamma + \delta\sqrt{3}$, where $\gamma, \delta \in \mathbb{Z}$, then $\gamma + \delta$ equals $\cdots$
    
    \item If $\int \cosec^5 x dx = \alpha \cot x \cosec x\brak{\cosec^2 x +\frac{3}{2}} + \beta\ln\abs{ \tan \frac{x}{2}} + C$, where $\alpha, \beta \in \mathbb{R}$, then the value of $8(\alpha + \beta)$ is:
    

    \item Let $f : \mathbb{R} \to \mathbb{R}$ be a thrice differentiable function such that $f\brak{0} = 0$, $f\brak{1} = 1$, $f\brak{2} = -1$, $f\brak{3} = 2$, and $f\brak{4} = -2$. Then, the minimum number of zeros of $3f\prime f\prime\prime+ff\prime\prime\prime$ is$\cdots$
    

    \item Let $A$ be a $2 \times 2$ symmetric matrix such that $A  \begin{bmatrix} 1 & 1 \end{bmatrix}=\begin{bmatrix} 3 & 7     \end{bmatrix}$, and the determinant of $A$ is $1$. If $A^{-1} = \alpha A + \beta I$, where $I$ is the identity matrix of order $2$, then $\alpha + \beta$ equals:
   
    \item Consider the function $f(x) = \frac{2x}{\sqrt{1 + 9x^2}}$. If the composition of $f\cdot\frac{\brak{f\cdot f \cdot f \cdots f}\brak{x}}{10 times}=\frac{2^{10}x}{\sqrt{1+9ax^2}}$ , then the value of $\sqrt{3a+1}$ is equal to $\cdots$
    

    \item Consider a line $L$ passing through points $P\brak{1, 2, 1}$ and $Q\brak{2, 1, -1}$. If the mirror image of point $A\brak{2, 2, 2}$ in the line $L$ is $\brak{\alpha, \beta, \gamma}$, then $\alpha + \beta + 6\gamma$ is$\cdots$
    

    \item In a tournament, a team plays $10$ matches with probabilities of winning and losing each match $\frac{1}{3}$ and $\frac{2}{3}$, respectively. Let $x$ be the number of matches that the team wins, and $y$ be the number of matches that the team loses. If the probability $P\brak{\abs{x - y} \leq 2}$ is $p$, then $3^9p$ equals$\cdots$
   

    \item Let $S = \{ \sin^2 2\theta:\brak{\sin^4 \theta +\cos^4
    \theta}x^2+\brak{ \sin2 \theta}x +\brak{\sin^6 \theta+  \cos^6\theta} = 0 \} has real root$. If $\alpha$ and $\beta$ are the smallest and largest elements of $S$, respectively, then $3((\alpha - 2)^2 + (\beta - 1)^2)$ equals$\cdots$
    



\end{enumerate}
\end{document}
