\documentclass[journal,12pt,twocolumn]{IEEEtran}

% Standard Packages
\usepackage{cite}
\usepackage{amsmath,amssymb,amsfonts,amsthm}
\usepackage{algorithmic}
\usepackage{graphicx}
\usepackage{textcomp}
\usepackage{xcolor}
\usepackage{listings}
\usepackage{enumitem}
\usepackage{mathtools}
\usepackage{gensymb}
\usepackage{comment}
\usepackage{tkz-euclide} 
\usepackage{gvv} 
\usepackage{longtable} 
\usepackage{calc} 
\usepackage{multirow}
\usepackage{hhline} 
\usepackage{tikz}
\usepackage{ifthen}
\usepackage{lscape}
\usepackage{tabularx}
\usepackage{float}

\renewcommand{\thetable}{\theenumi}
\theoremstyle{remark}

% Begin Document
\begin{document}

\bibliographystyle{IEEEtran}
\vspace{3cm}
\title{2012-AE-'53-65'}
\author{AI24BTECH11006 - Bugada Roopansha}
\maketitle

\begin{enumerate}[start=53]
 
    \item If the mass flow rate is $1 \frac{kg}{s}$, the power required to drive the compressor is
    \begin{enumerate}
        \item $50.5$ kW
        \item $40.5$ kW
        \item $30.5$ kW
        \item $20.5$ kW
    \end{enumerate}

    \textbf{Statement for Linked Answer Questions 54 and 55:} \\
    A thin-walled spherical vessel $\brak{1 \text{m inner diameter and} 10 \text{mm wall thickness}}$ is made of a material with $\sigma_y = 500$ MPa in both tension and compression.

    \item The internal pressure $p_y$, at yield, based on the von Mises yield criterion, if the vessel is floating in space, is approximately
    \begin{enumerate}
        \item $500$ MPa
        \item $250$ MPa
        \item $100$ MPa
        \item $20$ MPa
    \end{enumerate}

    \item If the vessel is evacuated $\brak{\text{internal pressure} =0}$ and subjected to external pressure, yielding according to the von Mises yield criterion \brak{\text{assuming elastic stability until yield}}
    \begin{enumerate}
        \item occurs at about half the pressure $p_y$
        \item occurs at about the same pressure $p_y$
        \item occurs at about double the pressure $p_y$
        \item never occurs.
    \end{enumerate}

    \item Choose the most appropriate alternative from the options given below to complete the following sentence: \\
   \textbf{ I \dots to have bought a diamond ring.}
    \begin{enumerate}
        \item have a liking
        \item should have liked
        \item would like
        \item may like
    \end{enumerate}

    \item Choose the most appropriate alternative from the options given below to complete the following sentence: \\
    \textbf{Food prices \dots again this month}.
    \begin{enumerate}
        \item have raised
        \item have been raising
        \item have been rising
        \item have arose
    \end{enumerate}

    \item Choose the most appropriate alternative from the options given below to complete the following sentence: \\
    \textbf{The administrators went on to implement yet another unreasonable measure, arguing that the measures were already \dots and one more would hardly make a difference.}
    \begin{enumerate}
        \item reflective
        \item utopian
        \item luxuriant
        \item unpopular
    \end{enumerate}

    \item Choose the most appropriate alternative from the options given below to complete the following sentence: \\
   \textbf{ To those of us who had always thought him timid, his \dots came as a surprise.}
    \begin{enumerate}
        \item intrepidity
        \item inevitability
        \item inability
        \item inertness
    \end{enumerate}

    \item The arithmetic mean of five different natural numbers is $12$. The largest possible value among the numbers is
    \begin{enumerate}
        \item $12$
        \item $40$
        \item $50$
        \item $60$
    \end{enumerate}

    \item Two policemen, A and B, fire once each at the same time at an escaping convict. The probability that A hits the convict is three times the probability that B hits the convict. If the probability of the convict not getting injured is $0.5$, the probability that B hits the convict is
    \begin{enumerate}
        \item $0.14$
        \item $0.22$
        \item $0.33$
        \item $0.40$
    \end{enumerate}

    \item The total runs scored by four cricketers P, Q, R, and S in years $2009$ and $2010$ are given in the following table:
    
    \begin{center}
    \begin{tabular}{|c|c|c|}
        \hline
        Player & 2009 & 2010 \\
        \hline
        P & 802 & 1008 \\
        Q & 765 & 912 \\
        R & 429 & 619 \\
        S & 501 & 701 \\
        \hline
    \end{tabular}
    \end{center}

    The player with the lowest percentage increase in total runs is
    \begin{enumerate}
        \item P
        \item Q
        \item R
        \item S
    \end{enumerate}

    \item If a prime number on division by $4$ gives a remainder of $1$, then that number can be expressed as
    \begin{enumerate}
        \item sum of squares of two natural numbers
        \item sum of cubes of two natural numbers
        \item sum of square roots of two natural numbers
        \item sum of cube roots of two natural numbers
    \end{enumerate}

    \item Two points $\brak{4, p}$ and $\brak{0, q}$ lie on a straight line having a slope of $\frac{3}{4}$. The value of $\brak{p - q}$ is
    \begin{enumerate}
        \item $-3$
        \item $0$
        \item $3$
        \item $4$
    \end{enumerate}

    \item \textbf{In the early nineteenth century, theories of social evolution were inspired less by Biology than by the conviction of social scientists that there was a growing improvement in social institutions. Progress was taken for granted and social scientists attempted to discover its laws and phases.} \\
    Which one of the following inferences may be drawn with the greatest accuracy from the above passage? \\
    Social scientists
    \begin{enumerate}
        \item did not question that progress was a fact.
        \item did not approve of Biology.
        \item framed the laws of progress.
        \item emphasized Biology over Social Sciences.
    \end{enumerate}



 
\end{enumerate}

\end{document}

