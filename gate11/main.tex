\documentclass[journal,12pt,twocolumn]{IEEEtran}

% Standard Packages
\usepackage{cite}
\usepackage{amsmath,amssymb,amsfonts,amsthm}
\usepackage{algorithmic}
\usepackage{graphicx}
\usepackage{textcomp}
\usepackage{xcolor}
\usepackage{listings}
\usepackage{enumitem}
\usepackage{mathtools}
\usepackage{gensymb}
\usepackage{comment}
\usepackage{tkz-euclide} 
\usepackage{gvv} 
\usepackage{longtable} 
\usepackage{calc} 
\usepackage{multirow}
\usepackage{hhline} 
\usepackage{tikz}
\usepackage{ifthen}
\usepackage{lscape}
\usepackage{tabularx}
\usepackage{float}

\renewcommand{\thetable}{\theenumi}
\theoremstyle{remark}

% Begin Document
\begin{document}

\bibliographystyle{IEEEtran}
\vspace{3cm}
\title{2017-PH-'27-39'}
\author{AI24BTECH11006 - Bugada Roopansha}
\maketitle

\begin{enumerate}[start=27]
 
    \item 
    An infinite solenoid carries a time-varying current $I\brak{t} = At^2$ with $A = 40$. The axis of the solenoid is along the $z$ direction. $\hat{r}$, $\hat{\theta}$, and $\hat{z}$ are the usual radial, polar, and axial directions in cylindrical polar coordinates. $B = B_r \hat{r} + B_\theta \hat{\theta} + B_z \hat{z}$ is the magnetic field at a point outside the solenoid. \\
    Which one of the following statements is true?
    \begin{enumerate}
        \item $B_r = 0, B_\theta = 0, B_z = 0$
        \item $B_r \neq 0, B_\theta = 0, B_z = 0$
        \item $B_r \neq 0, B_\theta \neq 0, B_z = 0$
        \item $B_r = 0, B_\theta = 0, B_z \neq 0$
    \end{enumerate}
    
    \item 

    A uniform volume charge density is placed inside a conductor $\brak{with resistivity 10^2 \, \Omega \text{m}}$. The charge density becomes $\frac{1}{\brak{2.718}}$ of its original value after time $t$ femtoseconds \brak{\text{up to two decimal places}}, with $\epsilon_0 = 8.854 \times 10^{-12} \, \frac{F}{m}$.
    
    \item 
    Water freezes at $0^\degree$C at atmospheric pressure $\brak{1.01 \times 10^5 \, \text{Pa}}$. The densities of water and ice at this temperature and pressure are $1000 \, \frac {kg}{m^3}$ and $934 \, \frac{kg}{m^3}$ respectively. The latent heat of fusion is $3.34 \times 10^5 \, \frac{J}{kg}$. The pressure required for depressing the melting temperature of ice by $1^\degree$C is  GPa \brak{\text{up to two decimal places}}.
    
    \item 
    The minimum number of NAND gates required to construct an OR gate is:
    \begin{enumerate}
        \item $2$
        \item $4$
        \item $5$
        \item $3$
    \end{enumerate}
  
    \item 
    Consider a 2-dimensional electron gas with a density of $10^{19} \, \text{m}^{-2}$. The Fermi energy of the system is  eV \brak{\text{up to two decimal places}}. \\
    Given: $m = 9.31 \times 10^{-31} \, \text{kg}$, $h = 6.626 \times 10^{-34} \, \text{Js}$, $e = 1.602 \times 10^{-19} \, \text{C}$
    
    \item
    The total energy of an inert-gas crystal is given by $E\brak{R} = \frac{0.5}{R^{12}} - \frac{1}{R^6}$ \brak{in eV}, where $R$ is the inter-atomic spacing in Angstroms. The equilibrium separation between the atoms is Angstroms \brak{\text{up to two decimal places}}.
    
    \item 
    Consider $N$ non-interacting, distinguishable particles in a two-level system at temperature $T$. The energies of the levels are $0$ and $\epsilon$, where $\epsilon > 0$. In the high temperature limit $\brak{k_B T \gg \epsilon}$, what is the population of particles in the level with energy $\epsilon$?
    \begin{enumerate}
        \item $\frac{N}{2}$
        \item $N$
        \item $\frac{N}{3}$
        \item $\frac{3N}{4}$
    \end{enumerate}
    
    \item 
    A free electron of energy $1$ eV is incident upon a one-dimensional finite potential step of height $0.75$ eV. The probability of its reflection from the barrier is \brak{\text{up to two decimal places}}.

    \item 
    Consider a one-dimensional potential well of width $3$ nm. Using the uncertainty principle $\brak{\Delta x \, \Delta p \geq \hbar/2}$, an estimate of the minimum depth of the well such that it has at least one bound state for an electron is \brak{\text{up to two decimal places}}. \\
    Given: $m_e = 9.31 \times 10^{-31} \, \text{kg}$, $h = 6.626 \times 10^{-34} \, \text{Js}$, $e = 1.602 \times 10^{-19} \, \text{C}$
    \begin{enumerate}
        \item $1$ $\mu$eV
        \item $1$ meV
        \item $1$ eV
        \item $1$ MeV
    \end{enumerate}
    
    \item 
    Consider a metal with free electron density of $6 \times 10^{22} \, \text{cm}^{-3}$. The lowest frequency of electromagnetic radiation to which this metal is transparent is $1.38 \times 10^{16} \, \text{Hz}$. If this metal had a free electron density of $1.8 \times 10^{23} \, \text{cm}^{-3}$ instead, the lowest frequency of electromagnetic radiation to which it would be transparent is $\times 10^{16}$ Hz \brak{\text{up to two decimal places}}.
    
    \item
    An object travels along the $x$-direction with velocity $\frac{c}{2}$ in a frame $O$. An observer in a frame $O\prime$ sees the same object travelling with velocity $\frac{c}{4}$. The relative velocity of $O\prime$ with respect to $O$ in units of $c$ is  \brak{\text{up to two decimal places}}.
    
    \item 
    The integral $\int \brak{x^2 - 1}^3 \, dx$ is equal to \brak{\text{up to two decimal places}}.

    \item 
    The imaginary part of an analytic complex function is $v\brak{x, y} = 2xy + 3y$. The real part of the function is zero at the origin. The value of the real part of the function at $1 + i$ is  \brak{\text{up to two decimal places}}.






 
\end{enumerate}

\end{document}

