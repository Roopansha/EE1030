%iffalse
\let\negmedspace\undefined
\let\negthickspace\undefined
\documentclass[journal,12pt,twocolumn]{IEEEtran}
\usepackage{cite}
\usepackage{amsmath,amssymb,amsfonts,amsthm}
\usepackage{algorithmic}
\usepackage{graphicx}
\usepackage{textcomp}
\usepackage{xcolor}
\usepackage{txfonts}
\usepackage{listings}
\usepackage{enumitem}
\usepackage{mathtools}
\usepackage{gensymb}
\usepackage{comment}
\usepackage[breaklinks=true]{hyperref}
\usepackage{tkz-euclide} 
\usepackage{listings}
\usepackage{gvv}                                        
%\def\inputGnumericTable{}    
\renewcommand{\thefigure}{\theenumi}
\renewcommand{\thetable}{\theenumi}
\usepackage[latin1]{inputenc}                                
\usepackage{color}                                            
\usepackage{array}                                            
\usepackage{longtable}                                       
\usepackage{calc}                                             
\usepackage{multirow}                                         
\usepackage{hhline}                                           
\usepackage{ifthen}                                           
\usepackage{lscape}
\usepackage{tabularx}
\usepackage{array}
\usepackage{float}
\newtheorem{theorem}{Theorem}[section]
\newtheorem{problem}{Problem}
\newtheorem{proposition}{Proposition}[section]
\newtheorem{lemma}{Lemma}[section]
\newtheorem{corollary}[theorem]{Corollary}
\newtheorem{example}{Example}[section]
\newtheorem{definition}[problem]{Definition}
\newcommand{\BEQA}{\begin{eqnarray}}
\newcommand{\EEQA}{\end{eqnarray}}
\theoremstyle{remark}
% Marks the beginning of the document
\begin{document}
\bibliographystyle{IEEEtran}
\vspace{3cm}
\title{2021-March Session-03-18-2021-shift-2}
\author{AI24BTECH11006 - Bugada Roopansha}
\maketitle
\section{SECTION - A}
\begin{enumerate}[start=16]
\item If P and Q are two statements, then which of the following compound statements is a tautology?
\begin{enumerate}
    \item 
\item  $\brak{\brak{P \Rightarrow Q} \land \neg Q} \Rightarrow P$ 
\item   $\brak{\brak{P \Rightarrow Q} \land \neg Q} \Rightarrow \neg P $
\item $ \brak{P \Rightarrow Q} \land \neg Q$ 
\item  $\brak{\brak{P \Rightarrow Q} \land \neg Q} \Rightarrow Q $


\end{enumerate}
\item  Consider a hyperbola H: $x^2 -2y^2=4$. Let the tangent at a point P $\brak{4,\sqrt{6}}$ meet the x-axis at Q and latus rectum at R $\brak{x_1,y_1} ,x_1>0$. If F is a focus of H which is nearer to the point P, then the area of $\triangle{QFR}$ is equal to:
\begin{enumerate}
\item $\sqrt{6} - 1$
\item  $4\sqrt{6} - 1$
\item  $4\sqrt{6} $
\item $\frac{7}{\sqrt{6}} - 2 $


\end{enumerate}
\item Let $f : \mathbb{R} \to \mathbb{R}$ be a function defined as 
\begin{equation}
       f(x) = 
\begin{cases} 
\frac{\sin((a+1)x) + \sin(2x)}{2x}, & $if $  x < 0 \\ 
b, & $if $ x = 0 \\ 
\frac{\sqrt{x + bx^{3} - \sqrt{x}}}{bx^{5/2}}, & $if $ x > 0 
\end{cases}
\end{equation}
. If f is continuous at $ x = 0$, then the value of a + b is equal to
\begin{enumerate}
    \item $-2$
    \item $\frac{-2}{5}$
    \item $\frac{-3}{2}$
    \item $-3$
\end{enumerate}
\item Let y=y$\brak{x}$ be the solution of the differential equation $\frac{dy}{dx}=\brak{y+1}\sbrak{\brak{y+1}e^{x^2/2}-x} ,0<x<2.1
$, with y$\brak{2}=0$. Then the value of $\frac{dy}{dx}$ at x=1 is equal to:
\begin{enumerate}
\item $\frac{e^{5/2}}{(1 + e^{2})^{2}} $
\item $ \frac{5 e^{1/2}}{(e^{2} + 1)^{2}} $
\item $ -\frac{2 e^{2}}{(1 + e^{2})^{2}} $
\item $ -\frac{e^{3/2}}{(e^{2} + 1)^{2}} $
\end{enumerate}
\item Let a tangent be drawn to the ellipse $\brak{x^2/27}+y^2=1$ at $\brak{3\sqrt{3} \cos \theta, \sin \theta} \quad $where $ \theta \in \brak{0, \frac{\pi}{2}}$.  Then the value of $\theta$ such that the sum of intercepts on axes made by a tangent is minimum is equal to:
\begin{enumerate}
    \item $\frac{\pi}{8}$
    \item $\frac{\pi}{6}$
    \item $\frac{\pi}{3}$
    \item $\frac{\pi}{4}$
\end{enumerate}
\section{SECTION - B}
\item Let P be a plane containing the line $
\frac{\sbrak{x - 1}}{3} = \frac{\sbrak{y + 6}}{4} = \frac{\sbrak{z + 5}}{2} $ and parallel to the line $
\frac{\sbrak{x - 3}}{4} = \frac{\sbrak{y - 2}}{-3} = \frac{\sbrak{z + 5}}{7} $. If the point $\brak{1,-1,\alpha}$ lies on the plane P, then the value of $|5\alpha|$ is equal to \dots
\item $
\sum_{r=1}^{10} r!( r^3 + 6r^2 + 2r + 5 ) = \alpha (11!)$ . Then the value of $\alpha$ is equal to \dots
\item The term independent of x in the expansion of $
\sbrak{ \frac{x + 1}{x^{2/3} - x^{1/3} + 1} - \frac{x - 1}{x - x^{1/2}} }^{10} , x \neq 1$ ,is equal to\dots
\item  Let $ \binom{n}{r} $ denote the binomial coefficient of $ x^r $ in the expansion of $ (1+x)^n $. If 
    $
    \sum_{k=0}^{10} [2^2 + 3k] \binom{n}{k} = \alpha \cdot 3^{10} + \beta \cdot 2^{10}
    $ 
then $\alpha + \beta$ is equal to\dots
\item  Let P $\brak{x}$ be a real polynomial of degree 3 which vanishes at x =- 3. Let P$\brak{x}$ have local minima at x = 1, local maxima at x = -1 and $
\int_{-1}^{1} P(x) \, dx = 18
$,then the sum of all the coefficients of the polynomial P $\brak{x}$ is equal to\dots
\item   Let the mirror image of the point $\brak{1, 3, a}$ with respect to the plane r. $\brak{2i - j + k} - b = 0$ be $\brak{0,-3, 5, 2}$. Then, the value of$ |a + b|$ is equal to\dots
\item If $f\brak{x}$ and $g\brak{x}$ are two polynomials such that the polynomial $P \brak{x} = f \brak{x^3} + x g \brak{x^3}$ is divisible by $^2 + x + 1$, then $P \brak{1}$ is equal to\dots
\item  Let I be an identity matrix of order $2 \times 2$  and $
P = \begin{bmatrix} 
2 & -1 \\ 
5 & -3 
\end{bmatrix}
$ . Then the value of $n\in N$ for which $P^n=5I-8P$ is equal to\dots
\item Let  $f : \mathbb{R} \to \mathbb{R}$ satisfy the equation 
$
f\brak{x + y} = f\brak{x} \cdot f\brak{y}$for all $x, y \in \mathbb{R} $and $ f\brak{x} \neq 0 $for any$ x \in \mathbb{R}
$. If the function f is differentiable at x = 0 and $f'\brak{0}=3$,then $
\lim_{h \to 0} \frac{1}{h}  \sbrak{f(h) - 1 }$ is equal to \dots
\item Let $y = y \brak{x}$ be the solution of the differential equation $x \, dy - y \, dx = \sqrt{x^2 - y^2} \, dx, x \geq 1
$ with $y\brak{1}=0$. If the area bounded by the line $x = 1, x = e^\pi, y = 0$ and $y = y(x)$ is$ \alpha e^{2\pi} + \beta $ then the value of $10 \brak{\alpha + \beta}$ is equal to \dots







\end{enumerate}
\end{document}
